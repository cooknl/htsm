% Options for packages loaded elsewhere
% Options for packages loaded elsewhere
\PassOptionsToPackage{unicode}{hyperref}
\PassOptionsToPackage{hyphens}{url}
\PassOptionsToPackage{dvipsnames,svgnames,x11names}{xcolor}
%
\documentclass[
  letterpaper,
  DIV=11,
  numbers=noendperiod]{scrartcl}
\usepackage{xcolor}
\usepackage[lmargin=1in,rmargin=1in,tmargin=1in,bmargin=1in]{geometry}
\usepackage{amsmath,amssymb}
\setcounter{secnumdepth}{5}
\usepackage{iftex}
\ifPDFTeX
  \usepackage[T1]{fontenc}
  \usepackage[utf8]{inputenc}
  \usepackage{textcomp} % provide euro and other symbols
\else % if luatex or xetex
  \usepackage{unicode-math} % this also loads fontspec
  \defaultfontfeatures{Scale=MatchLowercase}
  \defaultfontfeatures[\rmfamily]{Ligatures=TeX,Scale=1}
\fi
\usepackage{lmodern}
\ifPDFTeX\else
  % xetex/luatex font selection
\fi
% Use upquote if available, for straight quotes in verbatim environments
\IfFileExists{upquote.sty}{\usepackage{upquote}}{}
\IfFileExists{microtype.sty}{% use microtype if available
  \usepackage[]{microtype}
  \UseMicrotypeSet[protrusion]{basicmath} % disable protrusion for tt fonts
}{}
\makeatletter
\@ifundefined{KOMAClassName}{% if non-KOMA class
  \IfFileExists{parskip.sty}{%
    \usepackage{parskip}
  }{% else
    \setlength{\parindent}{0pt}
    \setlength{\parskip}{6pt plus 2pt minus 1pt}}
}{% if KOMA class
  \KOMAoptions{parskip=half}}
\makeatother
% Make \paragraph and \subparagraph free-standing
\makeatletter
\ifx\paragraph\undefined\else
  \let\oldparagraph\paragraph
  \renewcommand{\paragraph}{
    \@ifstar
      \xxxParagraphStar
      \xxxParagraphNoStar
  }
  \newcommand{\xxxParagraphStar}[1]{\oldparagraph*{#1}\mbox{}}
  \newcommand{\xxxParagraphNoStar}[1]{\oldparagraph{#1}\mbox{}}
\fi
\ifx\subparagraph\undefined\else
  \let\oldsubparagraph\subparagraph
  \renewcommand{\subparagraph}{
    \@ifstar
      \xxxSubParagraphStar
      \xxxSubParagraphNoStar
  }
  \newcommand{\xxxSubParagraphStar}[1]{\oldsubparagraph*{#1}\mbox{}}
  \newcommand{\xxxSubParagraphNoStar}[1]{\oldsubparagraph{#1}\mbox{}}
\fi
\makeatother

\usepackage{color}
\usepackage{fancyvrb}
\newcommand{\VerbBar}{|}
\newcommand{\VERB}{\Verb[commandchars=\\\{\}]}
\DefineVerbatimEnvironment{Highlighting}{Verbatim}{commandchars=\\\{\}}
% Add ',fontsize=\small' for more characters per line
\usepackage{framed}
\definecolor{shadecolor}{RGB}{241,243,245}
\newenvironment{Shaded}{\begin{snugshade}}{\end{snugshade}}
\newcommand{\AlertTok}[1]{\textcolor[rgb]{0.68,0.00,0.00}{#1}}
\newcommand{\AnnotationTok}[1]{\textcolor[rgb]{0.37,0.37,0.37}{#1}}
\newcommand{\AttributeTok}[1]{\textcolor[rgb]{0.40,0.45,0.13}{#1}}
\newcommand{\BaseNTok}[1]{\textcolor[rgb]{0.68,0.00,0.00}{#1}}
\newcommand{\BuiltInTok}[1]{\textcolor[rgb]{0.00,0.23,0.31}{#1}}
\newcommand{\CharTok}[1]{\textcolor[rgb]{0.13,0.47,0.30}{#1}}
\newcommand{\CommentTok}[1]{\textcolor[rgb]{0.37,0.37,0.37}{#1}}
\newcommand{\CommentVarTok}[1]{\textcolor[rgb]{0.37,0.37,0.37}{\textit{#1}}}
\newcommand{\ConstantTok}[1]{\textcolor[rgb]{0.56,0.35,0.01}{#1}}
\newcommand{\ControlFlowTok}[1]{\textcolor[rgb]{0.00,0.23,0.31}{\textbf{#1}}}
\newcommand{\DataTypeTok}[1]{\textcolor[rgb]{0.68,0.00,0.00}{#1}}
\newcommand{\DecValTok}[1]{\textcolor[rgb]{0.68,0.00,0.00}{#1}}
\newcommand{\DocumentationTok}[1]{\textcolor[rgb]{0.37,0.37,0.37}{\textit{#1}}}
\newcommand{\ErrorTok}[1]{\textcolor[rgb]{0.68,0.00,0.00}{#1}}
\newcommand{\ExtensionTok}[1]{\textcolor[rgb]{0.00,0.23,0.31}{#1}}
\newcommand{\FloatTok}[1]{\textcolor[rgb]{0.68,0.00,0.00}{#1}}
\newcommand{\FunctionTok}[1]{\textcolor[rgb]{0.28,0.35,0.67}{#1}}
\newcommand{\ImportTok}[1]{\textcolor[rgb]{0.00,0.46,0.62}{#1}}
\newcommand{\InformationTok}[1]{\textcolor[rgb]{0.37,0.37,0.37}{#1}}
\newcommand{\KeywordTok}[1]{\textcolor[rgb]{0.00,0.23,0.31}{\textbf{#1}}}
\newcommand{\NormalTok}[1]{\textcolor[rgb]{0.00,0.23,0.31}{#1}}
\newcommand{\OperatorTok}[1]{\textcolor[rgb]{0.37,0.37,0.37}{#1}}
\newcommand{\OtherTok}[1]{\textcolor[rgb]{0.00,0.23,0.31}{#1}}
\newcommand{\PreprocessorTok}[1]{\textcolor[rgb]{0.68,0.00,0.00}{#1}}
\newcommand{\RegionMarkerTok}[1]{\textcolor[rgb]{0.00,0.23,0.31}{#1}}
\newcommand{\SpecialCharTok}[1]{\textcolor[rgb]{0.37,0.37,0.37}{#1}}
\newcommand{\SpecialStringTok}[1]{\textcolor[rgb]{0.13,0.47,0.30}{#1}}
\newcommand{\StringTok}[1]{\textcolor[rgb]{0.13,0.47,0.30}{#1}}
\newcommand{\VariableTok}[1]{\textcolor[rgb]{0.07,0.07,0.07}{#1}}
\newcommand{\VerbatimStringTok}[1]{\textcolor[rgb]{0.13,0.47,0.30}{#1}}
\newcommand{\WarningTok}[1]{\textcolor[rgb]{0.37,0.37,0.37}{\textit{#1}}}

\usepackage{longtable,booktabs,array}
\usepackage{calc} % for calculating minipage widths
% Correct order of tables after \paragraph or \subparagraph
\usepackage{etoolbox}
\makeatletter
\patchcmd\longtable{\par}{\if@noskipsec\mbox{}\fi\par}{}{}
\makeatother
% Allow footnotes in longtable head/foot
\IfFileExists{footnotehyper.sty}{\usepackage{footnotehyper}}{\usepackage{footnote}}
\makesavenoteenv{longtable}
\usepackage{graphicx}
\makeatletter
\newsavebox\pandoc@box
\newcommand*\pandocbounded[1]{% scales image to fit in text height/width
  \sbox\pandoc@box{#1}%
  \Gscale@div\@tempa{\textheight}{\dimexpr\ht\pandoc@box+\dp\pandoc@box\relax}%
  \Gscale@div\@tempb{\linewidth}{\wd\pandoc@box}%
  \ifdim\@tempb\p@<\@tempa\p@\let\@tempa\@tempb\fi% select the smaller of both
  \ifdim\@tempa\p@<\p@\scalebox{\@tempa}{\usebox\pandoc@box}%
  \else\usebox{\pandoc@box}%
  \fi%
}
% Set default figure placement to htbp
\def\fps@figure{htbp}
\makeatother


% definitions for citeproc citations
\NewDocumentCommand\citeproctext{}{}
\NewDocumentCommand\citeproc{mm}{%
  \begingroup\def\citeproctext{#2}\cite{#1}\endgroup}
\makeatletter
 % allow citations to break across lines
 \let\@cite@ofmt\@firstofone
 % avoid brackets around text for \cite:
 \def\@biblabel#1{}
 \def\@cite#1#2{{#1\if@tempswa , #2\fi}}
\makeatother
\newlength{\cslhangindent}
\setlength{\cslhangindent}{1.5em}
\newlength{\csllabelwidth}
\setlength{\csllabelwidth}{3em}
\newenvironment{CSLReferences}[2] % #1 hanging-indent, #2 entry-spacing
 {\begin{list}{}{%
  \setlength{\itemindent}{0pt}
  \setlength{\leftmargin}{0pt}
  \setlength{\parsep}{0pt}
  % turn on hanging indent if param 1 is 1
  \ifodd #1
   \setlength{\leftmargin}{\cslhangindent}
   \setlength{\itemindent}{-1\cslhangindent}
  \fi
  % set entry spacing
  \setlength{\itemsep}{#2\baselineskip}}}
 {\end{list}}
\usepackage{calc}
\newcommand{\CSLBlock}[1]{\hfill\break\parbox[t]{\linewidth}{\strut\ignorespaces#1\strut}}
\newcommand{\CSLLeftMargin}[1]{\parbox[t]{\csllabelwidth}{\strut#1\strut}}
\newcommand{\CSLRightInline}[1]{\parbox[t]{\linewidth - \csllabelwidth}{\strut#1\strut}}
\newcommand{\CSLIndent}[1]{\hspace{\cslhangindent}#1}



\setlength{\emergencystretch}{3em} % prevent overfull lines

\providecommand{\tightlist}{%
  \setlength{\itemsep}{0pt}\setlength{\parskip}{0pt}}



 


\usepackage{scrlayer-scrpage}
\rohead{The Society of Flight Test Engineers}
\lofoot{SFTE 2025 International Symposium}    
\KOMAoption{captions}{tableheading}
\makeatletter
\@ifpackageloaded{tcolorbox}{}{\usepackage[skins,breakable]{tcolorbox}}
\@ifpackageloaded{fontawesome5}{}{\usepackage{fontawesome5}}
\definecolor{quarto-callout-color}{HTML}{909090}
\definecolor{quarto-callout-note-color}{HTML}{0758E5}
\definecolor{quarto-callout-important-color}{HTML}{CC1914}
\definecolor{quarto-callout-warning-color}{HTML}{EB9113}
\definecolor{quarto-callout-tip-color}{HTML}{00A047}
\definecolor{quarto-callout-caution-color}{HTML}{FC5300}
\definecolor{quarto-callout-color-frame}{HTML}{acacac}
\definecolor{quarto-callout-note-color-frame}{HTML}{4582ec}
\definecolor{quarto-callout-important-color-frame}{HTML}{d9534f}
\definecolor{quarto-callout-warning-color-frame}{HTML}{f0ad4e}
\definecolor{quarto-callout-tip-color-frame}{HTML}{02b875}
\definecolor{quarto-callout-caution-color-frame}{HTML}{fd7e14}
\makeatother
\makeatletter
\@ifpackageloaded{caption}{}{\usepackage{caption}}
\AtBeginDocument{%
\ifdefined\contentsname
  \renewcommand*\contentsname{Table of contents}
\else
  \newcommand\contentsname{Table of contents}
\fi
\ifdefined\listfigurename
  \renewcommand*\listfigurename{List of Figures}
\else
  \newcommand\listfigurename{List of Figures}
\fi
\ifdefined\listtablename
  \renewcommand*\listtablename{List of Tables}
\else
  \newcommand\listtablename{List of Tables}
\fi
\ifdefined\figurename
  \renewcommand*\figurename{Figure}
\else
  \newcommand\figurename{Figure}
\fi
\ifdefined\tablename
  \renewcommand*\tablename{Table}
\else
  \newcommand\tablename{Table}
\fi
}
\@ifpackageloaded{float}{}{\usepackage{float}}
\floatstyle{ruled}
\@ifundefined{c@chapter}{\newfloat{codelisting}{h}{lop}}{\newfloat{codelisting}{h}{lop}[chapter]}
\floatname{codelisting}{Listing}
\newcommand*\listoflistings{\listof{codelisting}{List of Listings}}
\makeatother
\makeatletter
\makeatother
\makeatletter
\@ifpackageloaded{caption}{}{\usepackage{caption}}
\@ifpackageloaded{subcaption}{}{\usepackage{subcaption}}
\makeatother
\usepackage{bookmark}
\IfFileExists{xurl.sty}{\usepackage{xurl}}{} % add URL line breaks if available
\urlstyle{same}
\hypersetup{
  pdftitle={Horizontal Time Safety Margin},
  colorlinks=true,
  linkcolor={blue},
  filecolor={Maroon},
  citecolor={Blue},
  urlcolor={Blue},
  pdfcreator={LaTeX via pandoc}}


\title{Horizontal Time Safety Margin}
\author{Nathan Cook \and Jackson Cook}
\date{}
\begin{document}
\maketitle


This work is licensed under
\href{https://creativecommons.org/licenses/by-nc-sa/4.0/}{CC BY-NC-SA
4.0}

\section*{Abstract}\label{sec-abstract}
\addcontentsline{toc}{section}{Abstract}

A technique is developed for establishing safety constraints during
ground test directional maneuvering. The technique is applicable to any
operation in an environment with hazards associated with departing a
prepared surface or violating proximity buffers around personnel or
property. The name ``Horizontal Time Safety Margin'' is inspired by the
Time Safety Margin (TSM) approach to establishing abort criteria for
highly dynamic vertical flight test maneuvering developed at Edwards
AFB, honoring the legacy of David ``Cools'' Cooley. Previous TSM work is
reviewed, and a general approach is provided to allow the specific
constraints of any system under test to be factored into implementation.
Considerations and lessons learned are discussed, based on recent taxi
testing.

\section*{Acronyms, Abbreviations, Symbols}\label{sec-acronyms}
\addcontentsline{toc}{section}{Acronyms, Abbreviations, Symbols}

\begin{longtable}[]{@{}
  >{\raggedright\arraybackslash}p{(\linewidth - 2\tabcolsep) * \real{0.2500}}
  >{\raggedright\arraybackslash}p{(\linewidth - 2\tabcolsep) * \real{0.7500}}@{}}
\toprule\noalign{}
\begin{minipage}[b]{\linewidth}\raggedright
Acronym
\end{minipage} & \begin{minipage}[b]{\linewidth}\raggedright
Definition
\end{minipage} \\
\midrule\noalign{}
\endhead
\bottomrule\noalign{}
\endlastfoot
TSM & Time Safety Margin \\
M\&S & Modeling and Simulation \\
\end{longtable}

\begin{longtable}[]{@{}
  >{\raggedright\arraybackslash}p{(\linewidth - 2\tabcolsep) * \real{0.2500}}
  >{\raggedright\arraybackslash}p{(\linewidth - 2\tabcolsep) * \real{0.7500}}@{}}
\toprule\noalign{}
\begin{minipage}[b]{\linewidth}\raggedright
Prepared Surface Symbol
\end{minipage} & \begin{minipage}[b]{\linewidth}\raggedright
Definition
\end{minipage} \\
\midrule\noalign{}
\endhead
\bottomrule\noalign{}
\endlastfoot
\(B\) & Distance buffer line \\
\(E\) & Edge of prepared surface \\
\(C\) & Centerline of prepared surface \\
\(w\) & Width between edges of prepared surface \\
\(o\) & Offset distance between centerline prepared surface and distance
buffer line \\
\(h\) & Halfwidth distance from centerline to edge of prepared
surface \\
\end{longtable}

\begin{longtable}[]{@{}
  >{\raggedright\arraybackslash}p{(\linewidth - 2\tabcolsep) * \real{0.2500}}
  >{\raggedright\arraybackslash}p{(\linewidth - 2\tabcolsep) * \real{0.7500}}@{}}
\toprule\noalign{}
\begin{minipage}[b]{\linewidth}\raggedright
Horizontal Time Safety Margin Parameter
\end{minipage} & \begin{minipage}[b]{\linewidth}\raggedright
Definitions
\end{minipage} \\
\midrule\noalign{}
\endhead
\bottomrule\noalign{}
\endlastfoot
\(b\) & Buffer distance between distance buffer line and edge of
prepared surface \\
\(s\) & Speed \\
\(\theta\) & Crossing angle \\
\(\theta_{abort}\) & Crossing angle value used as an abort criterion \\
\(t_{margin}\) or \(t\) & Time margin \\
\(r_{turn_{recovery}}\) or \(r\) & Turn radius during recovery \\
\end{longtable}

\begin{longtable}[]{@{}
  >{\raggedright\arraybackslash}p{(\linewidth - 2\tabcolsep) * \real{0.2500}}
  >{\raggedright\arraybackslash}p{(\linewidth - 2\tabcolsep) * \real{0.7500}}@{}}
\toprule\noalign{}
\begin{minipage}[b]{\linewidth}\raggedright
Derived Parameter
\end{minipage} & \begin{minipage}[b]{\linewidth}\raggedright
Definition
\end{minipage} \\
\midrule\noalign{}
\endhead
\bottomrule\noalign{}
\endlastfoot
\(\vec{v}\) & Velocity vector \\
\(d_{margin}\) & Distance margin, the product of time margin and
speed \\
\(\dot{\theta}\) & Angular rate \\
\(\ddot{\theta}\) & Angular acceleration \\
\(\dot{s}\) & Linear acceleration \\
\end{longtable}

\begin{longtable}[]{@{}
  >{\raggedright\arraybackslash}p{(\linewidth - 2\tabcolsep) * \real{0.2500}}
  >{\raggedright\arraybackslash}p{(\linewidth - 2\tabcolsep) * \real{0.7500}}@{}}
\toprule\noalign{}
\begin{minipage}[b]{\linewidth}\raggedright
Geometric Term
\end{minipage} & \begin{minipage}[b]{\linewidth}\raggedright
Definition
\end{minipage} \\
\midrule\noalign{}
\endhead
\bottomrule\noalign{}
\endlastfoot
\(\Delta\) & Triangle \\
\(\angle\) & Angle \\
\(I\) & Point of intersection of the ground path and \(B\) \\
\(T\) & Point where the recovery turn begins \\
\(\mu\) & Center of the turn circle \\
\(M\) & Point of intersection between \(B\) and a line perpendicular to
\(B\) that runs through \(\mu\) \\
\(x\) & Length of line segment from \(I\) to \(\mu\) \\
\(\alpha\) & Angle formed by \(\angle \mu IT\) \\
\(H\) & Point of intersection between \(B\) and a line perpendicular to
\(B\) that runs through \(T\) \\
\(F\) & Point where recovery turn is finished \\
\(V\) & Point of intersection between \(\overline{\mu F}\) and a line
perpendicular to \(\overline{\mu F}\) that runs through \(T\) \\
\end{longtable}

\newpage{}

\section{Introduction}\label{sec-intro}

The primary hazard under consideration in this paper is ``Departure from
the Prepared Surface,'' specifically in the context of low, medium, and
high speed taxi test. Following established practice, a mitigation for
this hazard is the establishment of abort criteria that allow for an
abort procedure to provide recovery of the system under test before
departing the prepared surface. The concept of horizontal time safety
margin resulted deliberations on how to determine the abort criteria
during taxi testing of new aircraft design as part of the build up to
first flight.

The lack of quantitative means of determining abort criteria prompted a
search for analogous methods that could inform the development of a
quantitative approach. Time safety margin for vertical aerial
maneuvering, developed to avoid collision with the surface, has many
features in common with the problem of avoiding departing the prepared
surface during horizontal ground maneuvering. Borrowing key elements
from time safety margin, a horizontal analogy was implemented. The
implementation is agnostic to any specific system and requires only
basic kinematics for initial estimates. Higher-fidelity estimates can be
generated by incorporating models of the system which include inertias
and moments, if desired and available.

\section{Time Safety Margin Origins}\label{sec-tsm_origins}

The history and development of time safety margin is well covered by
Bill ``Evil'' Gray in the 412th Test Wing Technical Information Handbook
titled ``Time Safety Margin: Theory And Practice.'' {[}1{]} Time safety
margin was born out of the 2009 fatal mishap of Dave ``Cools'' Cooley
during a diving test maneuver. The basic task of time safety margin is
to determine the time an aircraft may remain on its worst-case vector
until the planned recovery will no longer be sufficient to prevent a
mishap. The necessity for the development of time safety margin was
driven by large over- and under-estimations of maneuver setup and abort
criteria. The techniques prior to time safety margin were often
oversimplified and relied on imprecise or inaccurate assumptions.
Over-estimates lead to excessive setup climbs and too-conservative abort
calls, driving costs in time, fuel, and unnecessary repeated runs.
Under-estimates lead to insufficient margin and an increase in risk
exposure. After its incorporation into mandatory Air Force test
processes, time safety margin has continued to evolve as exposure to
multiple real-life scenarios has unveiled sensitivity to certain test
conditions.

Figure~\ref{fig-basic-dive} shows the essential components necessary to
understand and apply the time safety margin (TSM) concept. As defined in
{[}1{]}, a ``maneuver'' consists of a ``procedure'' and a ``recovery.''
In the figure, the procedure is simple upright pushover bunt dive, and
the recovery is an upright pull until level. The ``worst-case vector''
in this case is the lowest altitude with the steepest angle relative to
the ground. Extending the worst-case vector for a duration, the ``time
safety margin,'' generates a straight line, the ``TSM path,'' that ends
at the point where executing the remaining maneuver after the worst-case
vector results in a ``zero-spatial-margin'' recovery not impacting the
surface.

\begin{figure}

\centering{

\includegraphics[width=0.8\linewidth,height=\textheight,keepaspectratio]{images/tsm-theory-and-practice-fig-8.png}

}

\caption{\label{fig-basic-dive}From TSM Theory and Practice, Figure 8
``Basic Dive Recovery Terminology''}

\end{figure}%

The complexity of diving maneuvers makes closed-form solutions for
determining test safety margin untenable, so even the simplest scenario
with the :w most basic assumptions requires numerical methods to
simulate the unfolding recovery. Any general approach must consider
normal acceleration, true airspeed, Mach number, calibrated airspeed,
dynamic pressure, initial aircraft attitude, recovery technique,
asymmetric structural load, flight control system limiters, load
alleviation systems, engine control system augmentation, and
aircraft-specific factors. Time safety margin determination therefore
necessitates a software solution, and the most common implementations
have taken the form of MATLAB scripts and graphical user interfaces.

\section{Taxi Testing Requires Margins, Too}\label{sec-taxi_margin}

Taxi testing, consisting of ground maneuvers on prepared surfaces such
as ramps, taxiways, and runways, may at first glance not seem as
catastrophic in its consequences as diving at the ground. However,
departure from the prepared surface can result in total destruction of
an aircraft, loss of life, and destruction of adjacent property as much
as high-speed impact with the ground after an unsuccessful dive
recovery. Even minor damage or injury can set a test campaign back and
lead to unacceptable losses in schedule, cost, publicity, or trust. So,
while possibly not considered as dynamic or risky as aerial vertical
maneuvering, ground horizontal maneuvering requires margin, too.

The simplest margin is a buffer along the edge of the prepared surface,
providing a space-based abort criterion that can allow distance for a
recovery maneuver to remain on the prepared surface. But how to
determine the size of the buffer? What factors influence the balance
between too much buffer and too little buffer? As with diving time
safety margin, too much buffer leads to overly constrained maneuver and
nuisance abort calls and too little buffer leads to no real mitigation
of the hazard. Even before attempting to find that balance, a little
further consideration shows that a distance buffer alone is
insufficient.

The angle at which the aircraft crosses the buffer line greatly affects
the ability to recover and remain on the prepared surface. Crossing the
line at a right angle, heading directly toward the edge of the prepared
surface, offers far less margin than crossing at a shallow angle.
Therefore establishing a crossing angle that maintains margin for the
recovery maneuver must also enter the balance. Too shallow an angle
results in the same excessive margin as too large a distance buffer. Too
large an angle results in the same insufficient margin as too small a
distance buffer.

Intuitively, speed and turn radius also contribute to determining the
margin. Speed reduces the time available for the operator to react, and
can indirectly affect turn radius. Turn radius is constrained by forces.
Skidding puts an upper limit on force available to turn. Structural load
limits, the moment that initiates tipping, and shear induced on tires
put an upper limit on force allowable to turn. Turning forces are also
affected by surface conditions, tire wear, tire temperature, steering
kinematics, and steering dynamics.

\section{Horizontal Time Safety Margin Concept}\label{sec-htsm_concept}

How then, to determine the appropriate values of distance buffer and
crossing angle for a given speed and turn radius that result in a margin
that strikes the balance? The principles of time safety margin can be
brought to bear.

See Figure~\ref{fig-basic-turn} for the concept depicted from an
overhead view of a hypothetical runway. The figure is drawn to evoke
geometry of Figure~\ref{fig-basic-dive}, as this analogue was the
original inspiration for the concept of a horizontal time safety margin.
An aircraft is shown traveling from left to right, executing a right
turn toward the edge of the runway, then recovering with a left turn
with a smaller turn radius.{[}2{]} The aircraft is not-to-scale and
shown to assist in the visualization, but no actual aircraft
characteristics or dynamics are assumed or depicted. Additionally, the
aircraft is treated as a point mass for the purposes of this
illustration without considering main gear wheelbase or nose gear
geometry.

\begin{enumerate}
\def\labelenumi{\arabic{enumi}.}
\tightlist
\item
  Distance buffer

  \begin{itemize}
  \tightlist
  \item
    The distance buffer is a line, \(B\)
  \item
    \(B\) is parallel to the edge of the prepared surface, \(E\)
  \item
    \(B\) is also offset from the centerline, \(C\)
  \end{itemize}
\item
  Recovery maneuver

  \begin{itemize}
  \tightlist
  \item
    The recovery maneuver is a turn
  \item
    The turn is planned to begin at the moment of crossing the distance
    buffer line
  \item
    The turn is planned to end when the velocity vector is parallel to
    the edge of the prepared surface and no longer at risk of departing
  \end{itemize}
\item
  Worst-case vector

  \begin{itemize}
  \tightlist
  \item
    The worst-case vector is the velocity vector, \(\vec{v}\), at the
    moment of crossing the distance buffer line, \(B\)
  \item
    The vector consists of speed, \(s\), and direction, \(\theta\), the
    angle between the vector, \(\vec{v}\) and the buffer line, \(B\)
  \item
    The vector is not shown in Figure~\ref{fig-basic-turn} to avoid
    cluttering the image
  \end{itemize}
\item
  Time margin

  \begin{itemize}
  \tightlist
  \item
    The time margin is a duration
  \item
    The duration is the time for which the worst-case vector is extended
    prior to starting the recovery maneuver, shown as the dashed line in
    Figure~\ref{fig-basic-turn}.
  \end{itemize}
\end{enumerate}

\begin{figure}

\centering{

\includegraphics[width=0.8\linewidth,height=\textheight,keepaspectratio]{images/BigIdeaDiveCompare.png}

}

\caption{\label{fig-basic-turn}Basic Turn Recovery Inspired by TSM Basic
Dive Recovery}

\end{figure}%

\section{Horizontal Time Safety Margin
Implementation}\label{sec-htsm_implementation}

To implement the concept, a more detailed look at the geometry is in
order. Also, several additional simplifying assumptions are made for the
initial implementation:

\begin{enumerate}
\def\labelenumi{\arabic{enumi}.}
\tightlist
\item
  Constant speed during the full evolution
\item
  Circular turns with constant radii
\item
  Turn acceleration with no slip, skid, tip, or load exceedance
\item
  Symmetry for right and left turns
\item
  The speed and angle of the aircraft at the intersection of the ground
  path and the distance buffer line defines the worst-case vector, and
  therefore dynamics prior to the crossing can be ignored.
\end{enumerate}

With those assumptions in place, the following are given, as shown in
Figure~\ref{fig-given}:

\begin{enumerate}
\def\labelenumi{\arabic{enumi}.}
\tightlist
\item
  a prepared surface with

  \begin{enumerate}
  \def\labelenumii{\arabic{enumii}.}
  \tightlist
  \item
    straight edge, \(E\)
  \item
    a width, \(w\)
  \item
    a center\_line, \(C\)
  \item
    a buffer line parallel to the edge of the prepared surface, \(B\)
  \item
    a buffer distance, \(b\), from \(B\) to \(E\)
  \item
    an offset distance, \(o\), from \(C\) to \(B\)
  \item
    a halfwidth distance, \(h = \frac{w}{2}\), from \(C\) to \(E\)
  \end{enumerate}
\item
  a crossing vector, \(\vec{v}\), with magnitude, speed \(s\), and
  crossing angle \(\theta\)

  \begin{itemize}
  \tightlist
  \item
    The assumed constant speed, \(s\), is depicted by showing the
    vector, \(\vec{v}\) in three places

    \begin{enumerate}
    \def\labelenumii{\arabic{enumii}.}
    \tightlist
    \item
      With respect to the centerline, \(C\), with speed \(s\) and angle
      \(\theta\)
    \item
      With respect to the buffer line, \(B\), with speed \(s\) and angle
      \(\theta\)
    \item
      After full recovery, with speed \(s\) and parallel to the edge,
      \(E\)
    \end{enumerate}
  \end{itemize}
\item
  a desired time margin, \(t_{margin}\)
\item
  a turn radius, \(r_{turn_{recovery}}\)
\end{enumerate}

\begin{figure}

\centering{

\includegraphics[width=0.8\linewidth,height=\textheight,keepaspectratio]{images/BigIdeaStaticGiven.png}

}

\caption{\label{fig-given}The given parameters for horizontal time
safety margin}

\end{figure}%

\subsection{\texorpdfstring{Calculating the Maximum Recoverable Crossing
Angle,
\(\theta\)}{Calculating the Maximum Recoverable Crossing Angle, \textbackslash theta}}\label{sec-calc_theta}

The goal is to calculate the maximum crossing angle, \(\theta\), that
allows:

\begin{enumerate}
\def\labelenumi{\arabic{enumi}.}
\tightlist
\item
  straight line travel for

  \begin{enumerate}
  \def\labelenumii{\arabic{enumii}.}
  \tightlist
  \item
    time, \(t_{margin}\), and
  \item
    distance, \(d_{margin}\), where \(d_{margin} = t_{margin} \cdot s\)
  \end{enumerate}
\item
  then a recovery with radius \(r_{turn_{recovery}}\),
\item
  that results in the vehicle moving parallel to \(E\)
\item
  without departing the prepared surface
\end{enumerate}

Figure~\ref{fig-calc-examples} shows three angles that do not meet the
desired outcome for the given parameters. Speed, time margin, turn
radius, runway dimensions, and buffer distance are the same for all
three examples. One angle is too large for the given parameters.
Perpendicular to the buffer line, it results in departure from the
prepared surface. The other two are too small for the given parameters.
While they both result in a safe recovery, they are too shallow for the
time margin selected. The two shallow angles result in excessive margin
and potentially nuisance abort calls.

\begin{figure}

\centering{

\includegraphics[width=0.8\linewidth,height=\textheight,keepaspectratio]{images/BigIdeaStaticCalculate.png}

}

\caption{\label{fig-calc-examples}Three angles that result in
insufficient margin or excessive margin for the given speed, buffer
distance, turn radius and time margin}

\end{figure}%

To more precisely state the geometry problem, Figure~\ref{fig-angles}
zooms in to the buffer distance, \(b\). The figure depicts the geometry
for a recovery at the edge of the prepared surface after traveling a
straight-line distance for the full duration of the time safety margin.

\begin{figure}

\centering{

\includegraphics[width=0.8\linewidth,height=\textheight,keepaspectratio]{images/BigIdeaStaticAngles.png}

}

\caption{\label{fig-angles}Analysis of the desired crossing angle}

\end{figure}%

Upon inspection, two right triangles, \(\Delta \mu IT\) and
\(\Delta I\mu M\), can be constructed from the following elements:

\begin{itemize}
\tightlist
\item
  \(I\), the point of intersection between the ground path and the
  buffer line
\item
  \(T\), the point where the recovery turn begins
\item
  \(\mu\), the center of the turn circle
\item
  \(M\), the point of intersection between the buffer line and a line
  perpendicular to the buffer line that runs through \(\mu\)
\end{itemize}

The sides of the triangles can be related to the horizontal time safety
margin parameters:

\begin{itemize}
\tightlist
\item
  \(d_{margin}\), the length of line segment \(\overline{IT}\)
\item
  \(r_{turn_{recovery}}\), the length of line segment
  \(\overline{T\mu}\)
\item
  \(b - r_{turn_{recovery}}\), the length of line segment
  \(\overline{M\mu}\)
\end{itemize}

The two triangles share a hypotenuse,\(\overline{I\mu}\), with length
\(x\).

\(\overline{I\mu}\) divides the crossing angle, \(\theta\) into two
angles:

\begin{itemize}
\tightlist
\item
  \(\alpha\), the angle formed by \(\angle \mu IT\)
\item
  \(\theta - \alpha\), the angle formed by \(\angle MI\mu\)
\end{itemize}

The angle \(\alpha\) can be related to the horizontal time safety margin
parameters using the \(\tan\) of \(\Delta \mu IT\), as shown in
Equation~\ref{eq-alpha}.

\begin{equation}\phantomsection\label{eq-alpha}{
\alpha = \tan^{-1} \left( \frac{r_{turn_{recovery}}}{d_{margin}} \right)
}\end{equation}

The angle \(\theta - \alpha\) can be related to the horizontal time
safety margin parameters using the \(\sin\) of \(\Delta I\mu M\), as
shown in Equation~\ref{eq-theta_alpha_x}.

\begin{equation}\phantomsection\label{eq-theta_alpha_x}{
\theta - \alpha = \sin^{-1} \left( \frac{b - r_{turn_{recovery}}}{x}\right)
}\end{equation}

Making use of the Pythagorean Theorem from \(\Delta \mu IT\) defines
\(x\) in terms of \(r_{turn_{recovery}}\) and \(d_{margin}\), as shown
in Equation~\ref{eq-theta_alpha}.

\begin{equation}\phantomsection\label{eq-theta_alpha}{
\theta - \alpha = \sin^{-1} \left( \frac{b - r_{turn_{recovery}}}{\sqrt{{r_{turn_{recovery}}}^{2} + d_{margin}^{2}}} \right)
}\end{equation}

When adding the two equations, \(\alpha\) drops out and only \(\theta\)
remains, providing a closed-form solution for \(\theta\),
Equation~\ref{eq-theta}.

\begin{equation}\phantomsection\label{eq-theta}{
\theta = \tan^{-1} \left( \frac{r_{turn_{recovery}}}{d_{margin}} \right) + \sin^{-1} \left( \frac{b - r_{turn_{recovery}}}{\sqrt{{r_{turn_{recovery}}}^{2} + d_{margin}^{2}}} \right)
}\end{equation}

Substituting the speed, \(s\), and time margin, \(t_{margin}\), for the
distance, \(d_{margin}\), provides an equation completely in terms of
the horizontal time safety margin parameters,
Equation~\ref{eq-theta-full}.

\begin{equation}\phantomsection\label{eq-theta-full}{
\theta = \tan^{-1} \left( \frac{r_{turn_{recovery}}}{t_{margin} \cdot s} \right) + \sin^{-1} \left( \frac{b - r_{turn_{recovery}}}{\sqrt{{r_{turn_{recovery}}}^{2} + \left(t_{margin} \cdot s\right)^{2}}} \right)
}\end{equation}

\begin{tcolorbox}[enhanced jigsaw, coltitle=black, colbacktitle=quarto-callout-important-color!10!white, breakable, titlerule=0mm, leftrule=.75mm, opacitybacktitle=0.6, left=2mm, colframe=quarto-callout-important-color-frame, arc=.35mm, colback=white, bottomtitle=1mm, opacityback=0, toptitle=1mm, toprule=.15mm, bottomrule=.15mm, rightrule=.15mm, title=\textcolor{quarto-callout-important-color}{\faExclamation}\hspace{0.5em}{Important}]

This function provides the abort angle in radians, given that the
distance, time, and speed units are consistent in length and time. As
always, careful attention to unit conversions is necessary, especially
when working in typical aviation units, when working across teams, and
when working with international partners.

\end{tcolorbox}

Figure~\ref{fig-perfect} provides an example of a geometry using the
following values in normalized units as inputs:

\begin{itemize}
\tightlist
\item
  \(b = 4.0\),
\item
  \(r_{turn_{recovery}} = 1.0\),
\item
  \(s = 1.0\), and
\item
  \(t_{margin} = 4.0\).
\end{itemize}

This results in an angle of
\(\theta = 1.05981\mathrm{rad} = 60.7226^{\circ}\) and recovery of a
point mass without departing the prepared surface.

\begin{figure}

\centering{

\includegraphics[width=0.8\linewidth,height=\textheight,keepaspectratio]{images/BigIdeaStaticCalculatePerfect.png}

}

\caption{\label{fig-perfect}Demonstration of use of
Equation~\ref{eq-theta-full} to achieve a 4.0 time safety margin}

\end{figure}%

\subsection{Calculating the Remaining
Parameters}\label{sec-calc_remaining_params}

There may be scenarios in which the angle, \(\theta\), is already known
or defined. There may also be sensitivity studies in which the effect of
varying one parameter on the remaining parameters is of interest. In
these scenarios or studies, determining the other parameters may be
necessary to fully define or explore the trade space for the abort
criteria.

With a slight shift in perspective, the other parameters can be
determined, given a desired angle and the remaining parameters, as shown
in Figure~\ref{fig-others}. To declutter the figure:

\begin{itemize}
\tightlist
\item
  the turn radius, \(r_{turn_{recovery}}\), is shown as \(r\),
\item
  the time margin, \(t_{margin}\), is shown as \(t\), and
\item
  the time margin distance, \(d_{margin}\), is shown as the product
  \(t\cdot s\) .
\end{itemize}

\begin{figure}

\centering{

\includegraphics[width=0.8\linewidth,height=\textheight,keepaspectratio]{images/BigIdeaStaticOtherAngles.png}

}

\caption{\label{fig-others}A shift in geometry yields a closed form
solution for the other parameters, given the crossing angle, \(\theta\)}

\end{figure}%

\subsubsection{\texorpdfstring{Triangle
\(\Delta TIH\)}{Triangle \textbackslash Delta TIH}}\label{sec-triangle_tih}

A right triangle, \(\Delta TIH\), can be defined with

\begin{itemize}
\tightlist
\item
  vertices at points

  \begin{itemize}
  \tightlist
  \item
    \(I\), where the ground path intersects the buffer line, \(B\)
  \item
    \(T\), where the turn recovery starts
  \item
    \(H\), the point of intersection between \(B\) and a line
    perpendicular to \(B\) that runs through \(T\)
  \end{itemize}
\item
  angle, \(\angle HIT\), which is also the crossing angle, \(\theta\)
\item
  hypotenuse, \(\overline{IT}\), which is also the time margin distance,
  \(d_{margin} = t \cdot s\)
\item
  side, \(\overline{HT}\), which is opposite \(\theta\)
\end{itemize}

\begin{equation}\phantomsection\label{eq-sin-theta-ht}{
\sin\theta = \frac{\overline{HT}}{\overline{IT}} = \frac{\overline{HT}}{t \cdot s}
}\end{equation}

\(\overline{HT}\), can be expressed in terms of \(r\), and \(b\), with
the help of another triangle and line segment.

\subsubsection{\texorpdfstring{Triangle \(\Delta T\mu V\) and Line
Segment
\(\overline{M\mu}\)}{Triangle \textbackslash Delta T\textbackslash mu V and Line Segment \textbackslash overline\{M\textbackslash mu\}}}\label{sec-triangle_tmuv}

A right triangle, \(\Delta T\mu V\), can be defined with

\begin{itemize}
\tightlist
\item
  vertices at points

  \begin{itemize}
  \tightlist
  \item
    \(\mu\), the center point of the turn circle
  \item
    \(T\), where the turn recovery starts
  \item
    \(V\),

    \begin{itemize}
    \tightlist
    \item
      on the line segment \(\overline{\mu F}\)
    \item
      \(F\) is the point where the recovery turn is finished
    \item
      \(V\) is the point of intersection between \(\overline{\mu F}\)
      and a line perpendicular to \(\overline{\mu F}\) that runs through
      \(T\)\textbar{}
    \end{itemize}
  \end{itemize}
\item
  angle, \(\angle T\mu V\), which is also the crossing angle, \(\theta\)
\item
  hypotenuse, \(\overline{\mu T}\), which is also the recovery turn
  radius, \(r\)
\item
  side, \(\overline{\mu V}\), which is adjacent to \(\theta\), and
  therefore \(\overline{\mu V} = r\cos\theta\)
\end{itemize}

A line segment, \(\overline{M\mu}\), can be seen to be the difference
between the buffer distance, \(b\), and the recovery turn radius, \(r\),
thus \(\overline{M\mu} = b - r\).

\(\overline{HT}\) is equivalent to \(\overline{MV}\), the sum of
\(\overline{M\mu}\) and \(\overline{\mu V}\).

Therefore

\begin{equation}\phantomsection\label{eq-ht}{
\begin{aligned}
\overline{HT} = \overline{MV} = &&& \overline{M\mu} & + & \overline{\mu V}\\
= &&& \left( b - r\right) & + & \left(r\cos\theta\right) \\
= &&& b + r\left(\cos\theta - 1\right) &&
\end{aligned}
}\end{equation}

Combining Equation~\ref{eq-sin-theta-ht} and Equation~\ref{eq-ht} gives,

\begin{equation}\phantomsection\label{eq-all-params}{
\sin\theta = \frac{b + r \left(\cos\theta - 1 \right)}{t \cdot s}
}\end{equation}

\(\Delta TIH\) relates all five of the parameters, as shown in
Equation~\ref{eq-all-params}, with \(\theta\) the only parameter
included twice.\footnote{The fact that \(\theta\) appears twice in this
  equation within trigonometric functions is what prevents this equation
  from being used to solve for \(\theta\), thus the derivation in
  Section~\ref{sec-calc_theta}}

If \(\theta\) is known, then, the other four parameters can each be
solved simply in terms of the remaining parameters, as shown below.

\begin{equation}\phantomsection\label{eq-time}{
t = \frac{b + r \left(\cos\theta - 1 \right)}{s\sin\theta}
}\end{equation}

\begin{equation}\phantomsection\label{eq-speed}{
s = \frac{b + r \left(\cos\theta - 1 \right)}{t\sin\theta}
}\end{equation}

\begin{equation}\phantomsection\label{eq-buffer}{
b = t \cdot s\sin\theta + r \left(1 - \cos\theta \right)
}\end{equation}

\begin{equation}\phantomsection\label{eq-radius}{
r = \frac{t \cdot s\sin\theta - b }{ \cos\theta -1 }
}\end{equation}

Equation~\ref{eq-theta-full}, Equation~\ref{eq-time},
Equation~\ref{eq-speed}, Equation~\ref{eq-buffer}, and
Equation~\ref{eq-radius} provide a complete description of how to
determine the magnitudes of the buffer distance, \(b\), the time margin,
\(t\), the speed, \(s\), the turn radius, \(r\), and the crossing angle,
\(\theta\), that together provide an optimal recovery with no under- or
overshooting during recovery.

\section{Considerations}\label{sec-considerations}

The simplifying assumptions used to derive the equations above each have
additional considerations that will affect the utility of the horizontal
time safety margin concept. The section will address a few
considerations for each. As ever, safety planning and safe execution
rely on:

\begin{itemize}
\tightlist
\item
  sufficient knowledge of the system under test, including its
  capabilities, necessary constraints, and inherent margins,
\item
  the experience, training, currency, and proficiency of the test team
  in both system operation and test conduct,
\item
  test day conditions,
\item
  the regulatory and oversight systems in place for the planning and
  execution,
\item
  the time set aside for dedicated discussion, thought, and
  experimentation,
\item
  and the clarity, conciseness, correctness, and completeness of
  communication throughout planning and execution.
\end{itemize}

As such, no single paper or presentation can suffice for all
circumstances.

\subsection{\texorpdfstring{The Angle,
\(\theta\)}{The Angle, \textbackslash theta}}\label{sec-consider_angle}

\subsubsection{Angle Definition}\label{sec-consider_angle_defn}

The direction of the diving time safety margin worst-case vector,
\(\theta\), is treated in this paper as essentially the same as the
angle at which the vehicle crosses the the buffer line. This can be seen
as a somewhat circular, because the buffer line is, in part, established
as resulting from the worst-case vector. Which comes first, the angle or
the buffer? The decision to set the angle or the buffer distance first
is somewhat arbitrary, but can be influenced by the specific test
technique that is being employed and the test conditions that must be
met for the results to be valid, as well as the nature of the prepared
surface.

\subsubsection{\texorpdfstring{Angular Rate, \(\dot{\theta}\), and
Angular Acceleration,
\(\ddot{\theta}\)}{Angular Rate, \textbackslash dot\{\textbackslash theta\}, and Angular Acceleration, \textbackslash ddot\{\textbackslash theta\}}}\label{sec-consider_angle_rate}

The simplifying assumptions do not address angle rate at all, and in
fact the entire treatment assumes a point mass. Angular momentum of a
vehicle that is rotating toward the edge of the prepared surface must be
negated and in fact reversed. This requires torque, which in turn
requires reaction forces with the prepared surface, and which must be
transferred from the surface, through the physical geometry of the
vehicle. The extent and strength of any subcomponents and their
connections to the vehicle can impose additional constraints on the rate
and acceleration of the angle.

Additionally, the control system itself may have some constraint on the
rate or acceleration, either due to operator inputs via inceptor design,
control system logic via limiters or inherent properties, control system
implementation via feedback or estimations, or effector characteristics
such as flow rate limits, hardstops, or other nonlinearities.

\subsubsection{Proximity to the Centerline and the
Edge}\label{sec-consider_angle_prox}

Larger angles may be tolerated toward the center of the prepared surface
than toward the edge. A single abort angle \(\theta_{abort}\) is
typically applicable for a single combination of parameters, including
buffer distance. In a dynamic, real-time execution, a single
\(\theta_{abort}\) can actually lead to confusion and additional test
team workload to determine, in the moment, whether this is the region or
circumstance in which \(\theta_{abort}\) applies and whether an ``ABORT
ABORT ABORT Angle'' call is warranted. Multiple ``zones'' on the
prepared surface increase the complexity of test conduct and can result
in the very situations horizontal time safety margin is attempting to
mitigate, namely abort calls that are too soon, and therefore a
nuisance, or too late, and there for disastrous.

Test aids for visualizing the situation can assist, but themselves
become safety-critical products that must be validated and
version-controlled. Such aids could include:

\begin{itemize}
\tightlist
\item
  for a static \(\theta_{abort}\), regions of the prepared surface which
  are ``free-movement'', ``subject to \(\theta_{abort}\)'', and ``abort
  for proximity to edge''
\item
  a real-time (or faster-than-real-time) dynamically calculated
  \(\theta_{abort}\) that is compared with the current angle with the
  edge of the prepared surface, for a given minimum time safety margin
  value
\end{itemize}

\subsection{\texorpdfstring{The Speed,
\(s\)}{The Speed, s}}\label{sec-consider_speed}

\subsubsection{Speed Definition}\label{sec-consider_speed_defn}

The other component of the time safety margin worst-case, magnitude or
speed, \(s\), is treated in this paper in a similar fashion as the
angle, namely the speed at which the vehicle crosses the buffer line,
with the same considerations of the worst-case vector direction.

\subsubsection{\texorpdfstring{Linear Acceleration,
\(\dot{s}\)}{Linear Acceleration, \textbackslash dot\{s\}}}\label{sec-consider_speed_accel}

Also similar to the angle, the simplifying assumptions treat the speed
as constant throughout the time safety margin extension of the
worst-case vector, as well as during the constant turn radius recovery
maneuver. This is likely the largest deviation from actual execution, as
thrust, aerodynamic drag, rolling friction, and use of energy management
devices often result in speed changes.

Therefore, ``constant speed'' is the least likely assumption to hold,
but it is also the assumption most likely to result in actual margins
being greater than calculated. Simply by including any energy-decreasing
actions as part of the abort procedure, the speed can be expected to be
less than that used in the horizontal time safety margin calculation,
resulting in:

\begin{itemize}
\tightlist
\item
  more time available
\item
  excess buffer distance
\item
  more angle available
\item
  excess turn radius
\end{itemize}

Techniques such as ``Throttle - IDLE'' and ``Brakes - APPLY'' are fairly
common in many ground maneuvering tests, both of which contribute to
reducing the speed for some systems under certain conditions.

Conversely, any setup, test, or recovery techniques that could lead to
acceleration and an increase in speed would result in insufficient
margin.

\subsection{\texorpdfstring{The Buffer Distance,
\(b\)}{The Buffer Distance, b}}\label{sec-consider_buffer}

\subsubsection{Buffer Definition}\label{sec-consider_buffer_defn}

The buffer distance is the parameter most at the discretion of the test
team. It is also the simplest: a distance, perpendicular to the edge of
the prepared surface, forming a buffer line, beyond which testing will
cease and a recovery maneuver will commence. The simple definition can
mask the complexity of determining a useful buffer, and it may be
advisable to have the buffer distance fall out of the equations as one
of the last parameters whose values is set.

\subsubsection{Vehicle Geometry}\label{sec-consider_buffer_veh_geom}

The example above treats the vehicle as a point mass, but vehicles will
typically have 3 or more geometrically distributed points of contact
with the ground. Additionally, vehicles typically have portions of their
geometries that extend beyond the ground points of contact.

A main gear tire could end up off the prepared surface if track width,
defined as the lateral extent of ground contacts, is not considered,
even if the center of mass ``recovers'' at the edge of the prepared
surface. It is recommended to take half of the track width as an
additional offset in addition to the buffer distance to ensure all of
the ground contact points remain on the prepared surface.

Likewise, the nose gear could leave the prepared surface if wheel base,
defined as the longitudinal extent of ground contacts, is not
considered. It is recommended that a ground track for the nosewheel be
estimated to determine if additional offset is required for the test and
recovery conditions. This initial treatment of horizontal time safety
margin does not explore the conditions where nosewheel track becomes
critical, but intuition suggests that for shallower crossing angles
track width is more critical and for larger crossing angles wheel base
is more critical.

Similar considerations apply for the full vehicle geometry beyond the
extent of the ground points of contact, including nose, wings, and
empennage, to mitigate collision with objects or personnel in the
vertical dimension. Carefully survey test locations to account for three
dimensional collisions. An example might be taxi testing in proximity to
a hangar. Accounting for wheel base and track width will not be enough
to avoid collision without sufficient three-dimensional margin.

\subsection{\texorpdfstring{The Turn Radius,
\(r\)}{The Turn Radius, r}}\label{sec-consider_radius}

\subsubsection{Turn Radius Definition}\label{sec-consider_radius_defn}

Turn radius is the distance from the center of gravity to the
instantaneous center of the turn circle. The assumption is that this
radius is constant. This parameter is almost completely defined by
vehicle geometry and kinematics, but dynamics can also influence the
smallest acceptable turn radius.

\subsubsection{Geometry and Kinematics}\label{sec-consider_radius_geom}

The geometry of ground contact points with respect to the vehicle center
of mass, including wheel base fore and aft of the center of mass and the
track width right and left of the center of mass, describes the
kinematics of turning. Depending on the steering method for the vehicle,
there is typically a kinematically determined minimum turn radius. For
example, given nosewheel or tailwheel steering, there may be a maximum
steering angle available, which would determine the minimum turn radius.

\subsubsection{Dynamics}\label{sec-consider_radius_dynamics}

Turn radius is greatly affected by forces at the ground contact points
and at aerodynamic control surfaces. The means of steering can have a
significant effect both on the available turn radius and the constancy
of the turn radius.

The primary means of directional control on the ground are:

\begin{itemize}
\tightlist
\item
  Nosewheel steering
\item
  Differential braking
\item
  Rudder effector (or equivalent)
\item
  Differential thrust (if available)
\end{itemize}

The dynamics of each of these means vary greatly between systems, so
knowledge of the system under test is essential for safe and effective
safety planning and test conduct.

It is important to set aside time and provide resources for intentional
development of test and recovery techniques for the system under test.
Some techniques that work on one system may be ``negative training'' for
another system. For example, if differential thrust is the most
effective means of directional control, then ``Throttle - IDLE'' on all
available engines or on the incorrect engine may result in disaster.

\subsection{\texorpdfstring{The Time Margin,
\(t\)}{The Time Margin, t}}\label{sec-consider_time}

\subsubsection{Time Margin Definition}\label{sec-consider_time_defn}

Time margin is well-described in Section~\ref{sec-tsm_origins} and
Section~\ref{sec-htsm_concept}. The largest consideration of the
definition is determining, for each test and recovery maneuver
combination, the transition point out of the test manuever and into the
recovery maneuver, then padding that transition with the time margin.

\subsubsection{Duration Selection}\label{sec-consider_time_duration}

A default of 4 seconds time margin for ground maneuvering, based on the
work of Gray and the AGCAS team, is a solid starting point from which to
deviate. Complexity or simplicity of the test maneuver, the recovery
maneuver, and the transition between them can move the needle on time
margin duration. Readiness of the test team can also affect the
appropriate duration. The total available margin for test can also
factor in. A very large surface with very simple maneuvers has
inherently larger margins. A small surface with complex maneuvers has
inherently smaller margins.

For scenarios with inherently smaller margins, the Air Force Test Center
recommended mitigations for smaller time margin durations are a good
practice, as shown in Table~\ref{tbl-afman_tsm}.{[}3{]}

\begin{longtable}[]{@{}
  >{\raggedright\arraybackslash}p{(\linewidth - 12\tabcolsep) * \real{0.1429}}
  >{\raggedright\arraybackslash}p{(\linewidth - 12\tabcolsep) * \real{0.1429}}
  >{\raggedright\arraybackslash}p{(\linewidth - 12\tabcolsep) * \real{0.1429}}
  >{\raggedright\arraybackslash}p{(\linewidth - 12\tabcolsep) * \real{0.1429}}
  >{\raggedright\arraybackslash}p{(\linewidth - 12\tabcolsep) * \real{0.1429}}
  >{\raggedright\arraybackslash}p{(\linewidth - 12\tabcolsep) * \real{0.1429}}
  >{\raggedright\arraybackslash}p{(\linewidth - 12\tabcolsep) * \real{0.1429}}@{}}
\caption{Air Force Manual 11-2FTV3 TSM Risk
Assessment}\label{tbl-afman_tsm}\tabularnewline
\toprule\noalign{}
\begin{minipage}[b]{\linewidth}\raggedright
Short Title
\end{minipage} & \begin{minipage}[b]{\linewidth}\raggedright
TSM Range {[}sec{]}
\end{minipage} & \begin{minipage}[b]{\linewidth}\raggedright
Minimum Planning Fidelity
\end{minipage} & \begin{minipage}[b]{\linewidth}\raggedright
Recovery Procedure
\end{minipage} & \begin{minipage}[b]{\linewidth}\raggedright
Minimum Training \& Buildup
\end{minipage} & \begin{minipage}[b]{\linewidth}\raggedright
Recovery Initiation Call
\end{minipage} & \begin{minipage}[b]{\linewidth}\raggedright
Baseline Post-Mitigation Risk Level
\end{minipage} \\
\midrule\noalign{}
\endfirsthead
\toprule\noalign{}
\begin{minipage}[b]{\linewidth}\raggedright
Short Title
\end{minipage} & \begin{minipage}[b]{\linewidth}\raggedright
TSM Range {[}sec{]}
\end{minipage} & \begin{minipage}[b]{\linewidth}\raggedright
Minimum Planning Fidelity
\end{minipage} & \begin{minipage}[b]{\linewidth}\raggedright
Recovery Procedure
\end{minipage} & \begin{minipage}[b]{\linewidth}\raggedright
Minimum Training \& Buildup
\end{minipage} & \begin{minipage}[b]{\linewidth}\raggedright
Recovery Initiation Call
\end{minipage} & \begin{minipage}[b]{\linewidth}\raggedright
Baseline Post-Mitigation Risk Level
\end{minipage} \\
\midrule\noalign{}
\endhead
\bottomrule\noalign{}
\endlastfoot
Routine & \(t \ge 8\) & Normal ops & Routine & Not Required & Pilot &
Low \\
Focused & \(8 > t \ge 4\) & M\&S & Defined \& Documented & In-Flight
Buildup & Pilot & Low \\
Aided & \(4 > t \ge 2.5\) & Best Available M\&S & Defined \& Documented
& Sim Rehearsal \& In-Flight Buildup & Backup for Pilot & Low-Med \\
Redundantly Aided & \(2.5 > t \ge 1.5\) & Best Available M\&S & Defined
\& Documented & Sim Rehearsal \& In-Flight Buildup & Two Backups for
Pilot \& Anticipatory Cueing Desired & Med-High \\
Cued Anticipation & \(1.5 > t \ge 0\) & Best Available M\&S & Defined \&
Documented & Sim Rehearsal \& In-Flight Buildup & Two Backups for Pilot
\& Anticipatory Cueing Required & High \\
\end{longtable}

\subsection{Parameter Interactions}\label{sec-consider_interactions}

The above considerations are primarily single-parameter, however, there
are significant interactions between parameters.

\subsubsection{Speed and Turn Radius}\label{sec-consider_speed_radius}

Speed and turn radius are roughly proportional, but specific
relationships depend heavily on the system configuration and surface
conditions. In general, the higher the speed, the larger the turn
radius, and vice versa.

Skidding and slipping are primary factors in determining the
relationship between speed and turn radius. In turns, the higher the
speed the greater the lateral forces are exerted on ground contact
points, which can lead to skids. Skidding can also occur due to brake
torque increasing beyond rolling friction torque. Skidding is generally
to be avoided, as it makes recovery highly unpredictable, increases turn
radius, and can increase wear on tires.

Anti-skid braking systems can mitigate some skidding concerns when using
brakes for energy management or for directional control. But reliance on
anti-skid braking systems when planning for safety margin can result in
sudden loss of margin if the anti-skid braking system is mis-tuned or
malfunctions. Also, anti-skid braking systems can excite structural
vibration modes that expose the system to other hazards. A conservative
approach will assume no anti-skid braking system available, unless the
system is very mature and other contingency maneuvers are planned.

Structural load limits are another factor that causes turn radius to
increase with speed. Lateral loads are the source of the centripetal
acceleration towards the inside of the turn circle. The smaller the
radius and higher the speed, the larger the necessary force to keep the
turn constant.

It is possible for some systems to have minimum allowable turn radius be
a function of speed, which simplifies the modeling. In the absence of
such explicit relationships, it's important to realize that speed and
turn radius are not independent, and to cross-check turn radius values
with engineering judgement.

\subsubsection{Buffer Distance, Turn Radius, and
Angle}\label{sec-consider_interactions_buffer_radius_angle}

The interaction between buffer distance, turn radius, and angle is such
that, the shallower the angle, the larger the maximum possible turn
radius and the smaller the minimum possible buffer distance. As the
angle approaches perpendicular to the buffer line and the edge, the
buffer distance must be greater than the turn radius to allow for any
time margin. Thus, the greater the crossing angle, the greater the
buffer distance compared to the turn radius.

\subsubsection{Speed, Time Margin, and
Angle}\label{sec-consider_interactions_speed_time_angle}

The speed and time margin combine to become the distance margin
available. The larger the distance margin, the smaller the angle
available. Time margin should only be decreased for solid technical
reasons to meet high-value test objectives. Thus, the distance margin
will typically be primarily a result of speed, and test teams can expect
to have smaller abort angles at higher speeds.

\subsection{Beyond Horizontal Time Safety
Margin}\label{beyond-horizontal-time-safety-margin}

\subsubsection{Test Technique and Test Condition
Selection}\label{test-technique-and-test-condition-selection}

The best test point is the test point that doesn't need to be executed.
Live test with full-scale assets is expensive in many ways, and exposing
the test team, the system under test, and the people and property in the
vicinity of the test to the risks inherent in test operations should be
done with intention and with clear purpose. The next best test point is
the test point that is executed in a manner tailored to the system and
its environment, not one transferred from another system and another
context without customization. Horizontal time safety margin is a method
for planning and applying safety constraints to a test maneuver, but
careful thought should be applied to whether the test maneuver should be
included in the test campaign at all.

\subsubsection{``Longitudinal'' Time Safety
Margin?}\label{longitudinal-time-safety-margin}

The horizontal time safety margin developed here could be called
``lateral or yaw time safety margin,'' as it only is concerned with
recovery at the edge of the prepared surface parallel to the centerline.
The margin required to avoid departing the end of the prepared surface
in the longitudinal direction is not considered here. Contingencies in
the case of loss of energy management devices (brakes, anti-skid
systems, aerodynamic braking, etc) need to be considered, as well.

\subsubsection{Consequences of Experiencing the
Hazard}\label{consequences-of-experiencing-the-hazard}

Not all test limits are created equal. The transition between the
prepared surface and the surrounding environment relates directly to the
urgency and intensity of abort calls and procedures. A painted line on a
dry lakebed is not the same as a literal cliff adjacent to the prepared
surface, for example. Adjust test procedures and conduct to reflect the
severity of consequences should the hazard be encountered.

\section{Results During A Taxi Test Campaign}\label{sec-results}

The first applications of horizontal time safety margin were implemented
during low-speed taxi. Low-speed was defined as the speed region in
which aerodynamics had negligible contributions to directional control
or vertical forces. Initial testing was conducted prior to the
development of horizontal time safety margin. The first low-speed taxi
runs were executed on ramp spaces to avoid interfering with daily
general aviation operations. Buffer distances were established on the
ramp spaces and object collision was a higher hazard than departure from
the prepared surface. Once basic maneuverability and speed control was
confirmed, low-speed testing proceeded to fairly narrow taxiways,
approximately three to four times the track width. Taxiway buffer
distances were established as a function of track width and desired
distance from centerline. Angle offsets from centerline were small and
speeds were low enough that any brake application promptly stopped the
aircraft.

Low- and medium-speed taxi then progressed to runways. Medium-speed was
defined as the speed region in which the rudder had sufficient control
authority to be used for directional control but the wings did not
produce enough lift for unintentional liftoff and the elevator did not
have sufficient control authority to raise the nose gear off the ground.
Nosewheel steering was disabled during the transition from low- to
medium-speed taxi. Runway testing was the context in which horizontal
time safety margin was conceived. Test techniques included intentional
changes of heading to assess directional control margins. A previous
test campaign had put off directional control evaluations until
straight-line acceleration and braking were explored, but objectionable
directional control handling was discovered during contingency
operations, which is not the time to be discovering directional control
issues. Therefore, for this campaign, directional control was explored
early.

As the test team discussed appropriate abort criteria, the insufficiency
of buffer distance alone became apparent as larger heading offsets were
considered. As horizontal time safety margin was being initially
implements, buffer distances were held constant and abort angles were
calculated for each target test speed. At the time, the closed form
solution of Equation~\ref{eq-theta-full} had not been derived, so a
brute force spreadsheet optimization technique was used for the
calculations. Some basic assumptions about nosewheel skidding allowed
for a relationship between speed and minimum turn radius, so the minimum
turn radius became a function of speed and was not included separately.

For slower speeds, abort angle is very sensitive to changes in speed,
which meant keeping up with a large number of abort criteria for a given
set of test conditions. The complexity of updating abort angles for each
test condition became a distraction during test conduct. As speeds
increased, the calculated abort angles decreased less for each increment
that speed increased as the angle became less sensitive to speed at
higher speeds. The test team noticed that, since abort angles
consistently decreased with increasing speeds, an abort angle for a
given speed was conservative for all slower speeds. A ``tiered'' set of
abort angles was developed, making use of the change in sensitivity and
the fact that any abort angle at a given speed ``covered'' slower
speeds. A table of two tiers was developed, based on speed values
rounded up to the nearest 5 or 10 knot increment, such that a
``low-medium speed'' abort angle and a ``high speed'' abort angle for a
given test location sufficed.

At higher speeds, the abort angle reduced to small values, three to five
degrees, emphasizing to the test team that large directional control
inputs at higher speeds was cause for an abort, even before a heading
change developed. This was certainly the intuition of the test team
already after significant prior education, training, and experience, but
horizontal time safety margin provided a physical foundation for that
intuition, along with a tangible and calculable constraint based on that
foundation.

\section{Conclusions}\label{sec-conclusions}

Horizontal time safety margin provides a quantitative basis for safety
planning and test conduct of ground maneuvering test and recovery
techniques. With a few simplifying assumptions, a reasonable estimate
for buffer distances and abort angle criteria can be calculated for a
system under test executing a test technique under a test condition on a
given prepared surface. Horizontal time safety margin is appropriate for
ramp, taxiway, and runway testing from low- through high-speed taxi
testing.

The desired outcome of this paper and the accompanying presentation is
to solicit feedback on the approach and to increase fidelity where
appropriate. Additional considerations not covered here may surface
during implementation in other contexts.

\section*{Acknowledgments}\label{sec-ack}
\addcontentsline{toc}{section}{Acknowledgments}

\begin{itemize}
\tightlist
\item
  William ``Evil'' `Bill' Gray for not only the original contributions
  to the subject, but the thorough documentation of the context and
  discussion surrounding diving time safety margin.
\item
  Jackson Cook for providing a return on the investment into his
  undergraduate math degree, resulting in a closed-form solution for
  determining \(\theta\) based on a re-imagining of the trigonometry
  involved, which had eluded his father for more than a year.
  Figure~\ref{fig-angles} is his direct contribution, without which this
  paper would have suffered greatly.
\item
  Michael ``Ramjet'' `Mick' Mansfield for supporting this extension of
  prior practice into a new domain while under intense pressure to
  perform on a clean-sheet aircraft design.
\item
  The test pilots who put horizontal time safety margin through its
  paces: Kris ``WigB'' Rorberg, Tom ``Sally'' Fields, Jeremy ``NOVA''
  Vanderhal, Brett ``Pugs'' Pugsley, and David ``Pimp'' `Kees'
  Allamondola.
\item
  Madeleine ``Mad'' Graham, Mark ``SCIPR'' Jones, Tom ``Sulu'' Hill,
  NOVA, and Pimp for their insightful feedback.
\end{itemize}

\section*{References}\label{sec-ref}
\addcontentsline{toc}{section}{References}

\phantomsection\label{refs}
\begin{CSLReferences}{0}{0}
\bibitem[\citeproctext]{ref-Gray2016}
\CSLLeftMargin{{[}1{]} }%
\CSLRightInline{W. R. Gray, {``Time Safety Margin: Theory And
Practice.''} Sep. 2016. Available:
\url{https://apps.dtic.mil/sti/citations/AD1020034}}

\bibitem[\citeproctext]{ref-AirplanePNG}
\CSLLeftMargin{{[}2{]} }%
\CSLRightInline{{``Airplane Top View PNGs by Vecteezy.''} Available:
\url{https://www.vecteezy.com/free-png/airplane-top-view}}

\bibitem[\citeproctext]{ref-AFMAN11-2FTV3}
\CSLLeftMargin{{[}3{]} }%
\CSLRightInline{{``Flight Test Operations Procedures.''} Dec. 29, 2020.
Available:
\url{https://static.e-publishing.af.mil/production/1/af_a3/publication/afman11-2ftv3/afman11-2ftv3.pdf}}

\end{CSLReferences}

\section*{Biography}\label{sec-bio}
\addcontentsline{toc}{section}{Biography}

\textbf{Nathan ``CAP'N'' Cook}, SFTE Fellow and a United States Air
Force civil servant test engineer for 21 years, entered the startup
world in 2023, first as Principal Flight Test Engineer and Test
Conductor during Hermeus Quarterhorse Mk0 and Mk1 Taxi Test, and now as
Principal Flight Test Engineer at EpiSci testing sUAS collaborative
autonomy mesh networks and Group 5 flight and mission autonomy.

\textbf{Jackson Cook} graduated from the University of Florida in
December 2024 with a Bachelor's of Science in Mathematics and a minor in
Chinese. Topics of interest for him include algebra, topology, and
combinatorics. Currently, Jackson works as an educator at a science
center, tutors math, and does freelance video editing. He is also
collaborating in a project studying algebraic structures, such as
monoids, internal to graphs.

\newpage{}

\section*{Appendix A - Python Implementation}\label{sec-python}
\addcontentsline{toc}{section}{Appendix A - Python Implementation}

Equation~\ref{eq-theta-full}, Equation~\ref{eq-time},
Equation~\ref{eq-speed}, Equation~\ref{eq-buffer}, and
Equation~\ref{eq-radius} can be implemented in Python using the
\texttt{numpy} package for the trigonometric functions \texttt{arcsin},
\texttt{arctan}, and \texttt{sqrt}. These functions are defined with
arbitrary default values of buffer distance, \texttt{buffer}, recovery
turn radius, \texttt{radius}, speed, \texttt{speed}, and time safety
margin duration, \texttt{time}. These default values allow the functions
to return a value in the absence of one or more input argument
definitions.

\begin{Shaded}
\begin{Highlighting}[]
\ImportTok{from}\NormalTok{ numpy }\ImportTok{import}\NormalTok{ sin, cos, arcsin, arctan, sqrt}


\KeywordTok{def}\NormalTok{ abort\_angle(}\BuiltInTok{buffer}\OperatorTok{=}\FloatTok{4.0}\NormalTok{, radius}\OperatorTok{=}\FloatTok{1.0}\NormalTok{, speed}\OperatorTok{=}\FloatTok{1.0}\NormalTok{, time}\OperatorTok{=}\FloatTok{4.0}\NormalTok{):}
    \ControlFlowTok{return} \BuiltInTok{float}\NormalTok{(}
\NormalTok{        arctan(radius }\OperatorTok{/}\NormalTok{ (time }\OperatorTok{*}\NormalTok{ speed))}
        \OperatorTok{+}\NormalTok{ arcsin((}\BuiltInTok{buffer} \OperatorTok{{-}}\NormalTok{ radius) }\OperatorTok{/}\NormalTok{ sqrt(radius}\OperatorTok{**}\DecValTok{2} \OperatorTok{+}\NormalTok{ (time }\OperatorTok{*}\NormalTok{ speed) }\OperatorTok{**} \DecValTok{2}\NormalTok{))}
\NormalTok{    )}


\KeywordTok{def}\NormalTok{ abort\_buffer(angle}\OperatorTok{=}\NormalTok{abort\_angle(), radius}\OperatorTok{=}\FloatTok{1.0}\NormalTok{, speed}\OperatorTok{=}\FloatTok{1.0}\NormalTok{, time}\OperatorTok{=}\FloatTok{4.0}\NormalTok{):}
    \ControlFlowTok{return} \BuiltInTok{float}\NormalTok{(time }\OperatorTok{*}\NormalTok{ speed }\OperatorTok{*}\NormalTok{ sin(angle) }\OperatorTok{+}\NormalTok{ radius }\OperatorTok{*}\NormalTok{ (}\DecValTok{1} \OperatorTok{{-}}\NormalTok{ cos(angle)))}


\KeywordTok{def}\NormalTok{ time\_margin(}\BuiltInTok{buffer}\OperatorTok{=}\FloatTok{4.0}\NormalTok{, radius}\OperatorTok{=}\FloatTok{1.0}\NormalTok{, speed}\OperatorTok{=}\FloatTok{1.0}\NormalTok{, angle}\OperatorTok{=}\NormalTok{abort\_angle()):}
    \ControlFlowTok{return} \BuiltInTok{float}\NormalTok{(}
\NormalTok{        (}\BuiltInTok{buffer} \OperatorTok{+}\NormalTok{ radius }\OperatorTok{*}\NormalTok{ (cos(angle) }\OperatorTok{{-}} \DecValTok{1}\NormalTok{)) }\OperatorTok{/}\NormalTok{ (speed }\OperatorTok{*}\NormalTok{ sin(angle))}
\NormalTok{    )}


\KeywordTok{def}\NormalTok{ abort\_speed(}\BuiltInTok{buffer}\OperatorTok{=}\FloatTok{4.0}\NormalTok{, radius}\OperatorTok{=}\FloatTok{1.0}\NormalTok{, angle}\OperatorTok{=}\NormalTok{abort\_angle(), time}\OperatorTok{=}\FloatTok{4.0}\NormalTok{):}
    \ControlFlowTok{return} \BuiltInTok{float}\NormalTok{(}
\NormalTok{        (}\BuiltInTok{buffer} \OperatorTok{+}\NormalTok{ radius }\OperatorTok{*}\NormalTok{ (cos(angle) }\OperatorTok{{-}} \DecValTok{1}\NormalTok{)) }\OperatorTok{/}\NormalTok{ (time }\OperatorTok{*}\NormalTok{ sin(angle))}
\NormalTok{    )}


\KeywordTok{def}\NormalTok{ abort\_radius(}\BuiltInTok{buffer}\OperatorTok{=}\FloatTok{4.0}\NormalTok{, angle}\OperatorTok{=}\NormalTok{abort\_angle(), speed}\OperatorTok{=}\FloatTok{1.0}\NormalTok{, time}\OperatorTok{=}\FloatTok{4.0}\NormalTok{):}
    \ControlFlowTok{return} \BuiltInTok{float}\NormalTok{((time }\OperatorTok{*}\NormalTok{ speed }\OperatorTok{*}\NormalTok{ sin(angle) }\OperatorTok{{-}} \BuiltInTok{buffer}\NormalTok{) }\OperatorTok{/}\NormalTok{ (cos(angle) }\OperatorTok{{-}} \DecValTok{1}\NormalTok{))}
\end{Highlighting}
\end{Shaded}





\end{document}
